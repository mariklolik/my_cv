%-------------------------
% Resume in Latex
% Author : Shubhi Rani
% License : MIT
%------------------------

\documentclass[letterpaper,10.8pt]{article}

\usepackage{latexsym}
\usepackage[russian]{babel} 
\usepackage[empty]{fullpage}
\usepackage{titlesec}
\usepackage{marvosym}
\usepackage[usenames,dvipsnames]{color}
\usepackage{verbatim}
\usepackage{enumitem}
\usepackage[pdftex]{hyperref}
\usepackage{fancyhdr}


\pagestyle{fancy}
\fancyhf{} % clear all header and footer fields
\fancyfoot{}
\renewcommand{\headrulewidth}{0pt}
\renewcommand{\footrulewidth}{0pt}

% Adjust margins
\addtolength{\oddsidemargin}{-0.375in}
\addtolength{\evensidemargin}{-0.375in}
\addtolength{\textwidth}{1in}
\addtolength{\topmargin}{-.5in}
\addtolength{\textheight}{1in}

\urlstyle{rm}

\raggedbottom
\raggedright
\setlength{\tabcolsep}{0in}

% Sections formatting
\titleformat{\section}{
  \vspace{-3pt}\scshape\raggedright\large
}{}{0em}{}[\color{black}\titlerule \vspace{-5pt}]

%-------------------------
% Custom commands
\newcommand{\resumeItem}[2]{
  \item\small{
    \textbf{#1}{: #2 \vspace{-2pt}}
  }
}

\newcommand{\resumeItemWithoutTitle}[1]{
  \item\small{
    {\vspace{-2pt}}
  }
}

\newcommand{\resumeSubheading}[4]{
  \vspace{-1pt}\item
    \begin{tabular*}{0.97\textwidth}{l@{\extracolsep{\fill}}r}
      \textbf{#1} & #2 \\
      \textit{\small#3} & \textit{\small #4} \\
    \end{tabular*}\vspace{-5pt}
}

\newcommand{\resumeSubheadingexp}[2]{
  \vspace{-1pt}\item
    \begin{tabular*}{0.97\textwidth}{l@{\extracolsep{\fill}}r}
      \textbf{#1} & #2 \\
    \end{tabular*}\vspace{-5pt}
}

\newcommand{\resumeSubItem}[2]{\resumeItem{#1}{#2}\vspace{-4pt}}

\renewcommand{\labelitemii}{$\circ$}

\newcommand{\resumeSubHeadingListStart}{\begin{itemize}[leftmargin=*]}
\newcommand{\resumeSubHeadingListEnd}{\end{itemize}}
\newcommand{\resumeItemListStart}{\begin{itemize}}
\newcommand{\resumeItemListEnd}{\end{itemize}\vspace{-5pt}}

%-------------------------------------------
%%%%%%  CV STARTS HERE  %%%%%%%%%%%%%%%%%%%%%%%%%%%%


\begin{document}

%----------HEADING-----------------
\begin{tabular*}{\textwidth}{l@{\extracolsep{\fill}}r}
  \textbf{{\LARGE Каширский Марк}} & Email : \href{mailto:marshelo44@gmail.com}{marshelo44@gmail.com}\\
  \href{https://novosibirsk.hh.ru/resume/7dd6bcc0ff09c0c58a0039ed1f4263354f6d6c}{ХедХантер: https://novosibirsk.hh.ru/resume/7dd6bcc0ff09c0c58a0039ed1f4263354f6d6c} & Mobile : +7-983-186-7773 \\
  \href{https://github.com/mariklolik}{Github: https://github.com/mariklolik} \\
  \href{https://t.me/mariklolik}{Telegram: https://t.me/mariklolik}\\
\end{tabular*}

%-----------EDUCATION-----------------
\section{Образование}
  \resumeSubHeadingListStart
    \resumeSubheading
    {Яндекс.Лицей}{Барнаул}
      {Промышленное программирование на языке Python ;  Золотой сертификат об окончании}{Сентябрь 2017 - Август 2019}
       \resumeItemListStart
	    \resumeItem {Backend}{Библиотеки: Flask, Django}
        \resumeItem {Хранение и обработка данных}{SQLalchemy, MySQL, язык SQL}
        \resumeItem {Frontend}{SASS, Haml, Bootstrap}
        \resumeItem {Широкий опыт в разработке REST-API приложений}{}
        \resumeItem {Интерфейсы}{QT, Kivy, PyQt}
        \resumeItemListEnd

     %   \resumeSubheading
     %   %{MIPT Deep Learning School}{Moscow, Russia}
     % {Machine Learning and data processing on python ;  Bronze certificate of excellence}{Sep 2018 - Aug 2020}
     %  \resumeItemListStart
	 %   \resumeItem {Data visualization}{Seaborn}
     %   \resumeItem {Data preprocessing}{Pandas}
     %   \resumeItem {Machine learning}{Keras, Tensorflow, Scikit-Learn}
     %   \resumeItemListEnd


        \resumeSubheading
        {Школа Бэкенд Разработки Яндекса (ШБР)}{Яндекс, Удаленно}
      {Архитектура и разрботка бэкенда на python, тестирование, DevOps/Deploy; }{Сентябрь 2022 - Декабрь 2022}
       \resumeItemListStart
	    \resumeItem {Backend Development}{FastAPI, Tornado, Flask, Архитектура распределенных систем, микросервисный подход}
        \resumeItem {DevOps и Базы Данных}{ Docker compose, uvicorn, миграции ДБ, индексирование, архитектура БД. Реализации на postgres, Redis, MongoDB}
        \resumeItem {Тестирование}{Нагрзочное тестирование , stress, soak и volume тесты на PyTest, Debug-логирование.}
        \resumeItemListEnd
        
    \resumeSubheading
     {Новосибирский Государственный Университет}{Новосибирск}
      {Программная инженерия;  средний балл: 4.6}{Август 2021 - Август 2025}
      
	   {\scriptsize \textit{Пройденные курсы: Алгебра и геометрия,Информатика, Математический анализ, Дискретная математика, Математическая статистика, Теория вероятности, Математическая логика и теория алгоритмов, Программирование(C, C++, Java), Цифровая схемотехника}}
  \resumeSubHeadingListEnd

%
%--------PROGRAMMING SKILLS------------
\section{Навыки}
	\resumeSubHeadingListStart
	\resumeSubItem{Языки программирования}{Python, C++, Java, C, SQL, Unix scripting}
    \resumeSubItem{Фреймворки}{Flask, Django, Tornado, FastAPI, pytest, alembic, SqlAlchemy}
	\resumeSubItem{Инструменты}{Docker, GIT, Matlab, Postgres, Redis, MongoDB, Yandex Cloud, Yandex Tank, UNIX-based systems administration}
\resumeSubHeadingListEnd



%-----------EXPERIENCE-----------------
\section{Опыт}
\resumeSubHeadingListStart
    \resumeSubheadingexp
    {Яндекс.Лицей}{Барнаул}
    \resumeItemListStart
        \resumeItem
        {Веб-приложение на Flask}
          {Разработал сайт-аггрегатор новостей. Приложение отражает мои навыки FullStack-разработки. Frontend реализован при помощи HTML, HAML, SCSS, Bootstrap и Flask-Blueprints. Бэкенд написан на Flask.
          В качестве библиотеки для работы с БД используется SQLAlchemy.\ }\\
        \href{https://github.com/mariklolik/stpj}{Github проекта: https://github.com/mariklolik/stpj}
          \resumeItem{PyQt REST-Api проект}
          {Приложение расширяет возможности YandexMaps-API при помощи пользовательского интерфейса, позволяющего создавать собственные гео-скрипты. Использует PyQt как UI.}
          \resumeItem{Приложение для отслеживания погоды}{Разработал PyQt-приложение для помощи в составлении гардероба по погоде. Использует OpenWeatherMap-api и PyQt.}
          \resumeItem{Игра для голосового помощника Яндекс.Алиса}{Спроектировал и разработал приложение для умного помощника Яндекс.Алиса. Использовал Yandex-Alice api и SQLAlchemy для составления списка лидеров.}
          \resumeItem{Telgram, Discord, VK боты}{Разрабатывал различных ботов для облегчения взаимодействия бизнеса с конечным пользователем.}
    \resumeItemListEnd
    
    
    \resumeSubheadingexp
    {Школа Бэкенд Разработки Яндекса (ШБР) }{Яндекс, Удаленно}
    \resumeItemListStart
    \resumeItem{Сервис сокращения URL}{Спроектировал и задокументриовал приложение при помощи OpenAPI, реализовал фичу для роста бизнес-метрики и привлечения новых пользователей. Реализовано на FastAPI и uvicorn. Sqlalchemy с alembic в качестве DB и миграций }{}

    \resumeItem{Проектирование и внедерение микросервисного компонента}{Изучил проект фармацевтического магазина и создал микросервис для проверки корзины покупателя. Api реализовано на Flask, база данных postgre}{}

    \resumeItem{Работа с базами данных}{Осуществил миграции большой базы данных при помощи Alembic, спроектировал индексы для ускорения поиска в ней}{}

    \resumeItem{Тестирование}{Изучил Unit, Integration и System тестирование. Реализовывал различные виды тестов при помощи PyTest. Использовал Yandex Tank для нагрузочного тестирования}{}
    \resumeItemListEnd
    \resumeSubheadingexp
		{Новосибирский Государственный Университет}{Новосибирск}
		\resumeItemListStart
        \resumeItem{Язык C}
          {Работал над алгоритмами, изучаемыми на первом курсе: Калькулятор систем счисления, Алгоритм Бойера-Муура, алгоритм Хаффмана, алгоритмы на графах и прочие. Разрабатывал тесты для автоматической тестирующей системы, использовал CMake.}{}

        \resumeItem{Язык C++}
          {Изучил парадигмы ООП. Использовал их в написании работ:
          Дизайн и реализация типа данных Linked HashSet, 
          Создание интерпретатора языка Forth, 
          Дизайн и разрботка умных указателей. Работал с STL}{}
		\resumeItemListEnd

    
    \resumeItemListEnd
\resumeSubHeadingListEnd


\end{description}
%-------------------------------------------
\end{document}

